\documentclass{beamer}
\usepackage[T1]{fontenc}
\title{Presentatsiooni pealkiri}
\date{\today}
\author{Autori Nimi}
\institute{Asutus ja tiitel}

\def\logo{0_infosys_3lovi_est_cmyk.pdf}
\def\okthxbye{Aitäh!}
\def\epost{autori@epost.ee}

\usetheme{eegov}

\begin{document}

\begin{frame}
\titlepage
\end{frame}

\begin{frame}[t]
	\frametitle{See on alamleht}
	Nimekiri näeb välja nii:
	\begin{itemize}
		\item Esimene
		\item Teine
		\item Kolmas
		\begin{itemize}
			\item Esimene-esimene
			\item Teine-teine
			\item Kolmas-kolmas
		\end{itemize}
	\end{itemize}
\end{frame}

\begin{frame}[t]
	\frametitle{See on alamleht, millel on eriti pikk ühele reale mitte ära mahtuv pealkiri}
	Nimekirjad on olulised
	\begin{itemize}
		\item Sest
		\item Nad 
		\item Annavad 
		\item Struktuuri
	\end{itemize}
	Aga vahel saab ka ilma.
\end{frame}

\begin{frame}
	\frametitle{Pilt ütleb rohkem kui tuhat sõna}
	\includegraphics[width=\textwidth]{9413021187_9ea551669e_z.jpg}
	\vskip-1mm
	{\tiny Kain Kalju}
\end{frame}

\section{Alajaotuse pealkiri}

\begin{frame}[t]
	\frametitle{Ka slaide on mõistlik struktureerida}
	Tekst algab muide alati slaidi ülemisest servast, mitte keskelt
\end{frame}

\thnx
\end{document}
